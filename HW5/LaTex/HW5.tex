%% ---------------------------------------------------
%% Notice that we use the "report" class instead of "article"
%% ---------------------------------------------------
\documentclass{report}

\title{Project 5}
\author{\textbf{Jinhao Wei}}
\date{\textbf{19 February 2019}}

%% ---------------------------------------------------
%% CSBDformat specifies the format of our reports
%% ---------------------------------------------------
\usepackage{634format}

%% ---------------------------------------------------
%% enumerate 
%% ---------------------------------------------------
\usepackage{enumerate}

%% ---------------------------------------------------
%% listings is used for including our source code in reports
%% ---------------------------------------------------
\usepackage{listings}
\usepackage{textcomp}

%% ---------------------------------------------------
%% Packages for math environments
%% ---------------------------------------------------
\usepackage{amsmath}

%% ---------------------------------------------------
%% Packages for URLs and hotlinks in the table of contents
%% and symbolic cross references using \ref
%% ---------------------------------------------------
\usepackage{hyperref}

%% ---------------------------------------------------
%% Packages for using HOL-generated macros and displays
%% ---------------------------------------------------
\usepackage{holtex}
\usepackage{holtexbasic}
\input{commands}

\begin{document}


\newcommand{\HOLAllDate}{20 February 2019}
\newcommand{\HOLAllTime}{15:54}
\begin{SaveVerbatim}{HOLAllTheoremsconjSymThm}
\HOLTokenTurnstile{} \HOLFreeVar{p} \HOLSymConst{\HOLTokenConj{}} \HOLFreeVar{q} \HOLSymConst{\HOLTokenEquiv{}} \HOLFreeVar{q} \HOLSymConst{\HOLTokenConj{}} \HOLFreeVar{p}
\end{SaveVerbatim}
\newcommand{\HOLAllTheoremsconjSymThm}{\UseVerbatim{HOLAllTheoremsconjSymThm}}
\begin{SaveVerbatim}{HOLAllTheoremsconjSymThmAll}
\HOLTokenTurnstile{} \HOLSymConst{\HOLTokenForall{}}\HOLBoundVar{p} \HOLBoundVar{q}. \HOLBoundVar{p} \HOLSymConst{\HOLTokenConj{}} \HOLBoundVar{q} \HOLSymConst{\HOLTokenEquiv{}} \HOLBoundVar{q} \HOLSymConst{\HOLTokenConj{}} \HOLBoundVar{p}
\end{SaveVerbatim}
\newcommand{\HOLAllTheoremsconjSymThmAll}{\UseVerbatim{HOLAllTheoremsconjSymThmAll}}
\begin{SaveVerbatim}{HOLAllTheoremsproblemOneThm}
\HOLTokenTurnstile{} \HOLFreeVar{p} \HOLSymConst{\HOLTokenImp{}} (\HOLFreeVar{p} \HOLSymConst{\HOLTokenImp{}} \HOLFreeVar{q}) \HOLSymConst{\HOLTokenImp{}} (\HOLFreeVar{q} \HOLSymConst{\HOLTokenImp{}} \HOLFreeVar{r}) \HOLSymConst{\HOLTokenImp{}} \HOLFreeVar{r}
\end{SaveVerbatim}
\newcommand{\HOLAllTheoremsproblemOneThm}{\UseVerbatim{HOLAllTheoremsproblemOneThm}}
\newcommand{\HOLAllTheorems}{
\HOLThmTag{All}{conjSymThm}\HOLAllTheoremsconjSymThm
\HOLThmTag{All}{conjSymThmAll}\HOLAllTheoremsconjSymThmAll
\HOLThmTag{All}{problem1Thm}\HOLAllTheoremsproblemOneThm
}


%% --------------------------------------------------- the listings
%% parameter "language" is set to "ML"
%% ---------------------------------------------------
\lstset{language=ML}


\maketitle{}

\begin{abstract}
  This report is a summary on my solutions of problem \emph{8.4.1}, \emph{8.4.2} and\emph{8.4.3}. In these problems, we used \emph{HOL} to prove theorems and used \emph{EmitTex} to do pretty prints. Our works are contained in two folders respectively named \emph{LaTeX} and \emph{HOL}. In folder \emph{LaTeX}, we provided files that are used to produce this project report. In folder \emph{HOL}, we put all the proofs into one file named \emph{AllScript.sml} and used some other helper files to generate HOL reports named \emph{AllReport.pdf}. 
\end{abstract}

\begin{acknowledgments}
  This project follows the format and structure of \emph{sampleTheory} provided by Professor Shiu-Kai Chin. Specifically, this report is basically a modification on the file \emph{exampleEmitTeXReport.tex}, I merely saturated it up with my works.
\end{acknowledgments}

\tableofcontents{}

\chapter{Executive Summary}
\label{cha:executive-summary}

\textbf{All requirements for this project are satisfied}.  In
particular, we proved all the theorems in this project, pretty printed the HOL theories,
and made use of the \emph{EmitTeX} structure to typeset HOL theorems
in this report.

The following theorems are proved and their corresponding \LaTeX{}
macros used in this report. 
\begin{quote}
  \HOLAllTheorems
\end{quote}
However, due to my incompentence on pretty print techniques, the theorems' output order is not consistent with the theorems' proof order in the report.  

\chapter{Proof of problem1Thm}
\label{cha:proof-prob1theorem}


\section{Problem Statement}
\label{sec:problem-statement}
Our task is to prove the theorem \HOLAllTheoremsproblemOneThm


\section{HOL Code Proving problem1Thm}
\label{sec:hol-code-proving}


\begin{lstlisting}[frame=trBL]
val problem1Thm =
let
  val th1 = ASSUME``p:bool``
  val th2 = ASSUME``(p:bool)==>(q:bool)``
  val th3 = ASSUME``(q:bool) ==> (r:bool)``


  val th4 = MP th2 th1
  val th5 = IMP_TRANS th2 th3

  val th6 = MP th5 th1

  val th7 = DISCH(last(hyp(th6))) th6
  val th8 = DISCH(last(hyp(th7))) th7
in
  DISCH(last(hyp(th8))) th8
end;

\end{lstlisting}

\section{Session Transcript}
\label{sec:session-transcript}

\setcounter{sessioncount}{0}
\begin{session}
  \begin{scriptsize}
\begin{verbatim}

> > > > # # # # # # # # # ** types trace now on
> # # # # # # # # # ** Unicode trace now off
> val problem1Thm =
let
  val th1 = ASSUME``p:bool``
  val th2 = ASSUME``(p:bool)==>(q:bool)``
  val th3 = ASSUME``(q:bool) ==> (r:bool)``


  val th4 = MP th2 th1
  val th5 = IMP_TRANS th2 th3

  val th6 = MP th5 th1

  val th7 = DISCH(last(hyp(th6))) th6
  val th8 = DISCH(last(hyp(th7))) th7
in
  DISCH(last(hyp(th8))) th8
end;
# # # # # # # # # # # # # # # # val problem1Thm =
   |- (p :bool) ==> (p ==> (q :bool)) ==> (q ==> (r :bool)) ==> r:
   thm
> 
\end{verbatim}
  \end{scriptsize}
\end{session}


\chapter{Proof of conjSymThm}
\label{cha:proof-prob1theorem}


\section{Problem Statement}
\label{sec:problem-statement}
Our task is to prove the theorem \HOLAllTheoremsconjSymThm using \texttt{IMP_ANTISYM_RULE}.


\section{HOL Code Proving conjSymThm}
\label{sec:hol-code-proving}


\begin{lstlisting}[frame=trBL]
val conjSymThm =
let
  val tmp1Thm =
  	  let
		val thm1 = ASSUME``p/\q``
   		val thm2 = CONJUNCT1 thm1
   		val thm3 = CONJUNCT2 thm1
   		val thm4 = CONJ thm3 thm2
  	  in
		DISCH ``p/\q`` thm4
   	  end;
  val tmp2Thm =
      	  let
		val thm1 = ASSUME``q/\p``
  		val thm2 = CONJUNCT1 thm1
  		val thm3 = CONJUNCT2 thm1
  		val thm4 = CONJ thm3 thm2
	  in
		DISCH ``q/\p`` thm4
          end;	  
in        
  IMP_ANTISYM_RULE tmp1Thm tmp2Thm
end;
\end{lstlisting}

\section{Session Transcript}
\label{sec:session-transcript}

\setcounter{sessioncount}{0}
\begin{session}
  \begin{scriptsize}
\begin{verbatim}

> val conjSymThm =
let
  val tmp1Thm =
  	  let
		val thm1 = ASSUME``p/\q``
   		val thm2 = CONJUNCT1 thm1
   		val thm3 = CONJUNCT2 thm1
   		val thm4 = CONJ thm3 thm2
  	  in
		DISCH ``p/\q`` thm4
   	  end;
  val tmp2Thm =
      	  let
		val thm1 = ASSUME``q/\p``
  		val thm2 = CONJUNCT1 thm1
  		val thm3 = CONJUNCT2 thm1
  		val thm4 = CONJ thm3 thm2
	  in
		DISCH ``q/\p`` thm4
          end;	  
in        
  IMP_ANTISYM_RULE tmp1Thm tmp2Thm
end;
# # # # # # # # # # # # # # # # # # # # # # val conjSymThm =
   |- (p :bool) /\ (q :bool) <=> q /\ p:
   thm
\end{verbatim}
  \end{scriptsize}
\end{session}




\chapter{Proof of conjSymThmAll}
\label{cha:proof-demotheorem}

\section{Problem Statement}
\label{sec:problem-statement-1}

Our task is to prove the following theorem using \texttt{GENL}.

\HOLAllTheoremsconjSymThmAll



\section{HOL Code Proving conjSymThmAll}
\label{sec:hol-code-proving-1}

\begin{lstlisting}[frame=trBL]
val conjSymThmAll = GENL [``p:bool``, ``q:bool``] conjSymThm; 
\end{lstlisting}

\section{Session Transcript}
\label{sec:session-transcript-1}

\setcounter{sessioncount}{0}
\begin{session}
  \begin{scriptsize}
\begin{verbatim}

> val conjSymThmAll = GENL [``p:bool``, ``q:bool``] conjSymThm;
val conjSymThmAll =
   |- !(p :bool) (q :bool). p /\ q <=> q /\ p:
   thm
\end{verbatim}
  \end{scriptsize}
\end{session}

%% ------------------------------------------
%% Change to letters for appendix
%% ------------------------------------------

%% ------------------------------------------
%% this restarts the section numbering
%% ------------------------------------------
\appendix{} 


%% ------------------------------------------
% label using capital letters
%% ------------------------------------------
\renewcommand{\thechapter}{\Alph{chapter}} 

\chapter{Source Code for example1Script}
\label{cha:source-code-sample}

The following code is from \emph{AllScript.sml}, which is located
in a different subdirectory than this file.
\lstinputlisting{../HOL/AllScript.sml}



\end{document}
