\documentclass{report}


\usepackage{634format}
\usepackage{amsmath}

\usepackage{listings}
\usepackage{textcomp}

\usepackage{hyperref}

\usepackage{holtex}
\usepackage{holtexbasic}
\input{commands}

\begin{document}
\lstset{language = ML}


\title{Project 3}
\author{\textbf{Jinhao Wei}}
\date{Feburary 1, 2019}

\begin{document}
\maketitle{}

\chapter{Executive Summary}


\chapter{Exercise 4.6.3}
\label{cha:4.6.3}

\section{Problem Statement}
\label{sec:4.6.3-problem-statement}

In this problem, we will define several ML functions in two different ways. 

\section{Relevant Code}
\label{sec:4.6.3-relevant-code}

\section{Test Cases, Execution Transcripts and Explanations}
\label{sec:4.6.3-test-cases}
\subsection{Test cases}
\label{subsec:2.5.1-test-cases}

According to project requirements, we should run our functions on test cases below:
\begin{lstlisting}[frame=TB]

(******************************************************************************)
(* Test Cases                                                                 *)
(******************************************************************************)
timesPlus 100 27;
timesPlus 10 26;
timesPlus 1 25;
timesPlus 2 24;
timesPlus 30 23;
timesPlus 50 200;


\end{lstlisting}


\end{document}
