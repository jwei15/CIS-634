\documentclass{report}


\usepackage{634format}
\usepackage{amsmath}

\usepackage{listings}
\usepackage{textcomp}

\usepackage{hyperref}

\usepackage{holtex}
\usepackage{holtexbasic}
\input{commands}

\begin{document}
\lstset{language = ML}

\title{Project 2}
\author{\textbf{Jinhao Wei}}
\date{January 29, 2019}

\maketitle{}

\begin{abstract}
This project is mainly a practice on programming in ML. In this project, we conducted several basic operations, such as pattern matching, in HOL. This report is basically a summary of each of the practice problems. For each of the problems, this report contains:
\begin{itemize}
\item Problem Statements
\item Relevant Code
\item Test Results (and Analysis)
\end{itemize}



In this report, we used several packages for formatting the document, including:
\begin{itemize}
\item a style file for the course, \emph{634format.sty}
\item the \emph{listing} package for displaying and inputting ML source code
\item \emph{holtex.sty} and \emph{holtexbasic.sty}, which was provided in professor Shiu-Kai Chin's sample report
\end{itemize}

\end{abstract}

\begin{acknowledgments}
This project followed Professor Shiu-Kai Chin's sample documents' format, and used multiple \emph{.sty} files, such as \emph{634format.sty}, \emph{holtex.sty}, \emph{holtexbasic.sty}, in Professor Shiu-Kai Chin's sample report.
\end{acknowledgments}



\tableofcontents{}
\chapter{Executive Summary}
\label{cha:Executive-Summary}

\textbf{All requirements for this project are satisfied} Specifically,

\begin{description}
\item[Report Contents] \ \\
This report consists of:

\begin{enumerate}
\item[] Chapter~\ref{cha:Executive-Summary}: Executive Summary
\item[] Chapter~\ref{cha:exercise-2.5.1}: Exercise 2.5.1
    \begin{enumerate}
      \item[] Section~\ref{sec:2.5.1-problem-statement}: Problem Statement
      \item[] Section~\ref{sec:2.5.1-relevant-code}: Relevant Code
      \item[] Section~\ref{sec:2.5.1-test-results}: Test Results
    \end{enumerate}


\item[] Chapter~\ref{cha:exercise-3.4.1}
\begin{enumerate}
      \item[] Section~\ref{sec:3.4.1-problem-statement}: Problem Statement
      \item[] Section~\ref{sec:3.4.1-relevant-code}: Relevant Code
      \item[] Section~\ref{sec:3.4.1-test-results}: Test Results
    \end{enumerate}
\item[] Chapter~\ref{cha:exercise-3.4.2}
\begin{enumerate}
      \item[] Section~\ref{sec:3.4.2-problem-statement}: Problem Statement
      \item[] Section~\ref{sec:3.4.2-relevant-code}: Relevant Code
      \item[] Section~\ref{sec:3.4.2-test-results}: Test Results
    \end{enumerate}
     
\item[] Appendix~\ref{appendix:A}: Source Code For Exercise 2.5.1
\item[] Appendix~\ref{appendix:B}: Source Code For Exercise 3.4.1
\item[] Appendix~\ref{appendix:C}: Source Code For Exercise 3.4.2 
\end{enumerate}


\item [Reproducibility in ML and \LaTeX]\ \\
All ML and \LaTeX{} source files compile well on the environment provided by this course.
\end{description}

\chapter{Exercise 2.5.1}
\label{cha:exercise-2.5.1}

\section{Problem Statement}
\label{sec:2.5.1-problem-statement}

Our basic tasks in this problem are to:
\begin{enumerate}
\item Create a new \emph{.sml} file, which contains code and test cases
\item Evaluate values using the function defined in code
\end{enumerate}

\subsection{Functions to implement}
\label{subsec:2.5.1-functions-to-implement}

Our basic goal in this exercise is to implement and test a function named \emph{timesPlus}, which takes in two integer values and returns a pair of integer values.

\begin{align*}
timesPlus~x~y = (x \times y, x+y)
\end{align*}

\subsection{Test cases}
\label{subsec:2.5.1-test-cases}

According to project requirements, we should run our functions on test cases below:
\begin{lstlisting}[frame=TB]

(******************************************************************************)
(* Test Cases                                                                 *)
(******************************************************************************)
timesPlus 100 27;
timesPlus 10 26;
timesPlus 1 25;
timesPlus 2 24;
timesPlus 30 23;
timesPlus 50 200;


\end{lstlisting}


\section{Relevant Code}
\label{sec:2.5.1-relevant-code}

We will define our function \emph{timesPlus} as below:

\lstset{frameround = fttt}
\begin{lstlisting}[frame= trBL]
(*  Name: Jinhao Wei      *)
(*  Email: jwei15@syr.edu *)
fun timesPlus x y = (x*y, x+y);
\end{lstlisting}

\section{Test Results}
\label{sec:2.5.1-test-results}

If we send the above relevant code and test cases regions in the HOL, we will see a transcipt as below
\setcounter{sessioncount}{0}
\begin{session}
\begin{scriptsize}
\begin{verbatim}

---------------------------------------------------------------------
       HOL-4 [Kananaskis 11 (stdknl, built Sat Aug 19 09:30:06 2017)]

       For introductory HOL help, type: help "hol";
       To exit type <Control>-D
---------------------------------------------------------------------
> > > > # # val timesPlus = fn: int -> int -> int * int
> > # # # val it = (2700, 127): int * int
> val it = (260, 36): int * int
> val it = (25, 26): int * int
> val it = (48, 26): int * int
> val it = (690, 53): int * int
> val it = (10000, 250): int * int
> > 
*** Emacs/HOL command completed ***

> 

\end{verbatim}
\end{scriptsize}
\end{session}


\chapter{Exercise 3.4.1}
\label{cha:exercise-3.4.1}
\section{Problem Statement}
\label{sec:3.4.1-problem-statement}
In this problem, we focus on pattern matching in ML. We will use pattern matching to exact values from list and tuple. Our specific tasks are to produce expressions that

\begin{enumerate}
\item assign values to $listA$, a list of pairs
\item assign values to $e1B$, a pair, and $listB$, a list of pairs
\item assign values to $elC1$,  $elC1$, $elC2$, $elC3$, $elC4$ and $elC5$ as they are specified
\end{enumerate}

\section{Relevant Code}
\label{sec:3.4.1-relevant-code}
\lstset{frameround = ftttt, framexrightmargin = 0em}
\begin{lstlisting}[frame = trBL]
(****************************************************************************)
(*        Exercise: 3.4.1                                                   *)
(*        Author: Jinhao Wei                                                *)
(*        Date: January 29, 2019                                            *)
(****************************************************************************)
val listA = [(0,"Alice"), (1, "Bob"), (3, "Carol"), (4,"Dan")];
val elB::listB = listA;
val (elC1, elC2) = elB;
val (elC3::(elC4::(elC5::[]))) = listB;
\end{lstlisting}

\section{Test Results}
\label{sec:3.4.1-test-results}
If we send the above region to HOL,  we will see result as below:
\setcounter{sessioncount}{0}
\begin{session}
\begin{scriptsize}
\begin{verbatim}

> > > > val listA = [(0, "Alice"), (1, "Bob"), (3, "Carol"), (4, "Dan")]:
   (int * string) list
val elB = (0, "Alice"): int * string
val listB = [(1, "Bob"), (3, "Carol"), (4, "Dan")]: (int * string) list
val elC1 = 0: int
val elC2 = "Alice": string
val elC3 = (1, "Bob"): int * string
val elC4 = (3, "Carol"): int * string
val elC5 = (4, "Dan"): int * string
val it = (): unit
> 
*** Emacs/HOL command completed ***

\end{verbatim}
\end{scriptsize}
\end{session}

\chapter{Exercise 3.4.2}
\label{cha:exercise-3.4.2}


\section{Problem Statement}
\label{sec:3.4.2-problem-statement}
In this problem, we will set serveral values in emacs, and evaluate them respectively under HOL environment. Finally we will check and analyze the results.

\section{Relevant Code}
\label{sec:3.4.2-relevant-code}

\lstset{frameround = ftttt, framexrightmargin = 0em}
\begin{lstlisting}[frame = trBL]


(****************************************************************************)
(*  Exercise: 3.4.2                                                         *)
(*  Author: Jinhao Wei                                                      *)
(*  Date: January 29, 2019                                                  *)
(****************************************************************************)

val (x1,x2,x3) = (1,true,"Alice");
val pair1 = (x1,x3);
val list1 = [0,x1,2];
val list2 = [x2,x1];
(*The line above should fail*)
(*x2 and x1 have different data type so they don't form a list*)

val list3 = (1 :: [x3]);
(*The line above should fail*)
(*int value can't be inserted to the front of a char list*)
\end{lstlisting}

\section{Test Results and Analysis}
\label{sec:3.4.2-test-results}

According to the requirement, we should evaluate the values one by one and check the results. Therefore, I did not send the whole region to HOL, I ran each of them one after another.

The first three evaluation was quite successful. The first case is just a simple pattern matching. The second case is a tuple which can receive any different data types. As for the third case, all items in the list have the same data type, so the evaluation was successful.
\setcounter{sessioncount}{0}
\begin{session}
\begin{scriptsize}
\begin{verbatim}

> > > > val (x1,x2,x3) = (1,true,"Alice");
val x1 = 1: int
val x2 = true: bool
val x3 = "Alice": string
> 
> val pair1 = (x1,x3);
val pair1 = (1, "Alice"): int * string
> 
> val list1 = [0,x1,2];
val list1 = [0, 1, 2]: int list
> 

\end{verbatim}
\end{scriptsize}
\end{session}

\newpage
The following case failed to be evaluated, this is because all elements in a list should have the same data type. Howerver, from previous calculation we know that $x1$ is of type \emph{int} while $x2$ is \emph{bool}. So they can not make a list.
\begin{session}
\begin{scriptsize}
\begin{verbatim}

> val list2 = [x2,x1];
poly: : error: Elements in a list have different types.
   Item 1: x2 : bool
   Item 2: x1 : int
   Reason:
      Can't unify bool (*In Basis*) with int (*In Basis*)
         (Different type constructors)
Found near [x2, x1]

\end{verbatim}
\end{scriptsize}
\end{session}


The following case also failed, this is because $1$ is a data of type \emph{int} while $[x3]$ is a list of \emph{string}. Therefore the calculation failed.  
\begin{session}
\begin{scriptsize}
\begin{verbatim}

> 
> val list3 = (1 :: [x3]);
poly: : error: Type error in function application.
   Function: :: : int * int list -> int list
   Argument: (1, [x3]) : int * string list
   Reason:
      Can't unify int (*In Basis*) with string (*In Basis*)
         (Different type constructors)
Found near (1 :: [x3])
Static Errors

\end{verbatim}
\end{scriptsize}
\end{session}


\appendix{}
\chapter{Source Code for Exercise 2.5.1}
\label{appendix:A}
The following code is from \emph{ex-2-5-1.sml}
\lstinputlisting{./ML/ex-2-5-1.sml}
\chapter{Source Code for Exercise 3.4.1}
\label{appendix:B}
The following code is from \emph{ex-3-4-1.sml}
\lstinputlisting{./ML/ex-3-4-1.sml}
\chapter{Source Code for Exercise 3.4.2}
\label{appendix:C}
The following code is from \emph{ex-3-4-2.sml}
\lstinputlisting{./ML/ex-3-4-2.sml}
\end{document}