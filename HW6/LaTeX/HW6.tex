%% ---------------------------------------------------
%% Notice that we use the "report" class instead of "article"
%% ---------------------------------------------------
\documentclass{report}

\title{Project 6}
\author{\textbf{Jinhao Wei}}
\date{\textbf{27 February 2019}}

%% ---------------------------------------------------
%% CSBDformat specifies the format of our reports
%% ---------------------------------------------------
\usepackage{634format}

%% ---------------------------------------------------
%% enumerate 
%% ---------------------------------------------------
\usepackage{enumerate}

%% ---------------------------------------------------
%% listings is used for including our source code in reports
%% ---------------------------------------------------
\usepackage{listings}
\usepackage{textcomp}

%% ---------------------------------------------------
%% Packages for math environments
%% ---------------------------------------------------
\usepackage{amsmath}

%% ---------------------------------------------------
%% Packages for URLs and hotlinks in the table of contents
%% and symbolic cross references using \ref
%% ---------------------------------------------------
\usepackage{hyperref}

%% ---------------------------------------------------
%% Packages for using HOL-generated macros and displays
%% ---------------------------------------------------
\usepackage{holtex}
\usepackage{holtexbasic}
\input{commands}

\begin{document}


\input{../HOL/HOLReports/HOLexerciseNine}
\newcommand{\HOLexerciseOneZeroDate}{27 February 2019}
\newcommand{\HOLexerciseOneZeroTime}{23:34}
\begin{SaveVerbatim}{HOLexerciseOneZeroTheoremsproblemOneXXthm}
\HOLTokenTurnstile{} \HOLFreeVar{M} \HOLFreeVar{s}
\end{SaveVerbatim}
\newcommand{\HOLexerciseOneZeroTheoremsproblemOneXXthm}{\UseVerbatim{HOLexerciseOneZeroTheoremsproblemOneXXthm}}
\begin{SaveVerbatim}{HOLexerciseOneZeroTheoremsproblemTwoXXthm}
\HOLTokenTurnstile{} \HOLFreeVar{p} \HOLSymConst{\HOLTokenImp{}} \HOLSymConst{\HOLTokenNeg{}}\HOLFreeVar{q}
\end{SaveVerbatim}
\newcommand{\HOLexerciseOneZeroTheoremsproblemTwoXXthm}{\UseVerbatim{HOLexerciseOneZeroTheoremsproblemTwoXXthm}}
\begin{SaveVerbatim}{HOLexerciseOneZeroTheoremsproblemThreeXXthm}
\HOLTokenTurnstile{} \HOLFreeVar{r} \HOLSymConst{\HOLTokenDisj{}} \HOLFreeVar{s}
\end{SaveVerbatim}
\newcommand{\HOLexerciseOneZeroTheoremsproblemThreeXXthm}{\UseVerbatim{HOLexerciseOneZeroTheoremsproblemThreeXXthm}}
\newcommand{\HOLexerciseOneZeroTheorems}{
\HOLThmTag{exercise10}{problem1_thm}\HOLexerciseOneZeroTheoremsproblemOneXXthm
\HOLThmTag{exercise10}{problem2_thm}\HOLexerciseOneZeroTheoremsproblemTwoXXthm
\HOLThmTag{exercise10}{problem3_thm}\HOLexerciseOneZeroTheoremsproblemThreeXXthm
}


%% --------------------------------------------------- the listings
%% parameter "language" is set to "ML"
%% ---------------------------------------------------
\lstset{language=ML,breaklines}


\maketitle{}

\begin{abstract}
  This report is a summary on my solutions of problem \emph{9.5.1}, \emph{9.5.2}, \emph{9.5.3}, \emph{10.4.1}, \emph{10.4.2} and \emph{10.4.3}. In these problems, we used \emph{HOL} to prove theorems and used \emph{EmitTex} to do pretty prints. Our works are contained in two folders respectively named \emph{LaTeX} and \emph{HOL}. In folder \emph{LaTeX}, we provided files that are used to produce this project report. In folder \emph{HOL}, we put all the proofs into two seperate files named \emph{exercise9Script.sml} and \emph{exercise10Script.sml}. We also used some other helper files to generate HOL reports named \emph{exercise9Report.pdf} and \emph{exercise10Report.pdf}. 
\end{abstract}

\begin{acknowledgments}
  This project follows the format and structure of \emph{sampleTheory} provided by Professor Shiu-Kai Chin. To make it more accurate, this project mostly followed the format of my last project, which is project 5, and project 5 followed the sturcture of Professor Shiu-Kai Chin's \emph{sampleTheory} project. Besides, my friend Yihao Sun helped me in solving problem \emph{10.4.2}. It was by his hint that I realized how this problem should be solved.
\end{acknowledgments}

\tableofcontents{}

\chapter{Executive Summary}
\label{cha:executive-summary1}

\textbf{This is a late submission}
Due to my incompentence on HOL's tactic-based proof techniques, this is a late submission. 
\\

\textbf{All requirements for this project are satisfied}.  In
particular, we proved all the theorems in this project, pretty printed the HOL theories,
and made use of the \emph{EmitTeX} structure to typeset HOL theorems
in this report.

The following theorems are proved and their corresponding \LaTeX{}
macros used in this report. 
\begin{quote}
  \HOLexerciseNineTheorems
  \HOLexerciseOneZeroTheorems
\end{quote}


\chapter{Proof of absorptionRule}


\section{Problem Statement}

Our task is to prove the theorem \HOLexerciseNineTheoremsabsorptionRule without using \texttt{PROVE_TAC}.


\section{HOL Code Proving absorptionRule}


\begin{lstlisting}[frame=trBL]
val absorptionRule =
TAC_PROOF(
([], ``!p q.(p ==> q) ==> p ==> (p /\ q)``),
( REPEAT STRIP_TAC THEN
  RES_TAC THEN
  ASM_REWRITE_TAC [])
);

\end{lstlisting}

\section{Session Transcript}
\label{sec:session-transcript}

\setcounter{sessioncount}{0}
\begin{session}
  \begin{scriptsize}
\begin{verbatim}
 
> # # # # # # val absorptionRule =
    [] |- !(p :bool) (q :bool). (p ==> q) ==> p ==> p /\ q:
   thm
\end{verbatim}
  \end{scriptsize}
\end{session}


\chapter{Proof of constructiveDilemmaRule}
\label{cha:proof-prob1theorem}


\section{Problem Statement}
\label{sec:problem-statement}
Our task is to prove the theorem \HOLexerciseNineTheoremsconstructiveDilemmaRule without using \texttt{PROVE_TAC}.


\section{HOL Code Proving constructiveDilemmaRule}
\label{sec:hol-code-proving}


\begin{lstlisting}[frame=trBL]
val constructiveDilemmaRule =
TAC_PROOF(
([], ``!p q r s. (p==>q)/\(r==>s) ==> p\/r ==> q\/s``),
REPEAT STRIP_TAC THENL
[(DISJ1_TAC THEN RES_TAC) , (DISJ2_TAC THEN RES_TAC)]
);
\end{lstlisting}

\section{Session Transcript}
\label{sec:session-transcript}

\setcounter{sessioncount}{0}
\begin{session}
  \begin{scriptsize}
\begin{verbatim}

> # # # # # val constructiveDilemmaRule =
    []
|- !(p :bool) (q :bool) (r :bool) (s :bool).
     (p ==> q) /\ (r ==> s) ==> p \/ r ==> q \/ s:
   thm
\end{verbatim}
  \end{scriptsize}
\end{session}


\chapter{Proof of absorptionRule2}


\section{Problem Statement}
Our task is to prove the theorem \HOLexerciseNineTheoremsabsorptionRuleTwo using \texttt{PROVE_TAC}.


\section{HOL Code Proving absorptionRule2}


\begin{lstlisting}[frame=trBL]
val absorptionRule2 =
TAC_PROOF(([], ``!p q.(p ==> q) ==> p ==> (p /\ q)``),PROVE_TAC[]);
\end{lstlisting}

\section{Session Transcript}
\label{sec:session-transcript}

\setcounter{sessioncount}{0}
\begin{session}
  \begin{scriptsize}
\begin{verbatim}
 
> # Meson search level: ........
val absorptionRule2 =
    [] |- !(p :bool) (q :bool). (p ==> q) ==> p ==> p /\ q:
   thm
> 
\end{verbatim}
  \end{scriptsize}
\end{session}

\chapter{Proof of constructiveDilemmaRule2}

\section{Problem Statement}

Our task is to prove the following theorem using \texttt{PROVE_TAC}.

\HOLexerciseNineTheoremsconstructiveDilemmaRuleTwo


\section{HOL Code Proving constructiveDilemmaRule2}

\begin{lstlisting}[frame=trBL]
val constructiveDilemmaRule2 =
TAC_PROOF(
([], ``!p q r s. (p==>q)/\(r==>s) ==> p\/r ==> q\/s``),
PROVE_TAC[]
);
\end{lstlisting}

\section{Session Transcript}
\label{sec:session-transcript-1}


\setcounter{sessioncount}{0}
\begin{session}
  \begin{scriptsize}
\begin{verbatim}

> # # # # Meson search level: ................
val constructiveDilemmaRule2 =
    []
|- !(p :bool) (q :bool) (r :bool) (s :bool).
     (p ==> q) /\ (r ==> s) ==> p \/ r ==> q \/ s:
   thm
> 
\end{verbatim}
  \end{scriptsize}
\end{session}



\chapter{Proof of problem1_thm}

\section{Problem Statement}

Our task is to prove the following theorem without using \texttt{PROVE_TAC}.

\HOLexerciseOneZeroTheoremsproblemOneXXthm


\section{HOL Code Proving problem1_thm}

\begin{lstlisting}[frame=trBL]
val problem1_thm =
TAC_PROOF(
([``!x:'a. P(x)==>M(x)``, ``(P:'a->bool)(s:'a)``], ``(M:'a->bool)(s:'a)``),
RES_TAC
);
\end{lstlisting}

\section{Session Transcript}


\setcounter{sessioncount}{0}
\begin{session}
  \begin{scriptsize}
\begin{verbatim}

> # # # # val problem1_thm =
   
[(P :'a -> bool) (s :'a),
 !(x :'a). (P :'a -> bool) x ==> (M :'a -> bool) x]
|- (M :'a -> bool) (s :'a):
   thm
> 
\end{verbatim}
  \end{scriptsize}
\end{session}


\chapter{Proof of problem2_thm}

\section{Problem Statement}

Our task is to prove the following theorem without using \texttt{PROVE_TAC}.

\HOLexerciseOneZeroTheoremsproblemTwoXXthm


\section{HOL Code Proving problem1_thm}

\begin{lstlisting}[frame=trBL]
val problem2_thm =
TAC_PROOF(
([``p/\q ==> r``, ``r==>s``, ``~s``], ``p==>~q``),
REPEAT STRIP_TAC THEN
REPEAT RES_TAC
);
\end{lstlisting}

\section{Session Transcript}


\setcounter{sessioncount}{0}
\begin{session}
  \begin{scriptsize}
\begin{verbatim}

> # # # # # val problem2_thm =
   
[~(s :bool), (r :bool) ==> (s :bool),
 (p :bool) /\ (q :bool) ==> (r :bool)] |- (p :bool) ==> ~(q :bool):
   thm
\end{verbatim}
  \end{scriptsize}
\end{session}


\chapter{Proof of problem3_thm}

\section{Problem Statement}

Our task is to prove the following theorem without using \texttt{PROVE_TAC}.

\HOLexerciseOneZeroTheoremsproblemThreeXXthm


\section{HOL Code Proving problem3_thm}

\begin{lstlisting}[frame=trBL]
val problem3_thm =
TAC_PROOF(
([``~(p/\q)``, ``~p ==> r``, ``~q ==> s``],``r\/s``),

PAT_ASSUM ``~(p/\q)`` (fn th => (ASSUME_TAC (REWRITE_RULE [DE_MORGAN_THM] th))) THEN

PAT_ASSUM ``~p\/~q`` (fn th => ASSUME_TAC (REWRITE_RULE [] (DISJ_IMP th))) THEN

ASSUME_TAC(IMP_TRANS(ASSUME ``p==>~q``)(ASSUME ``~q==>s``)) THEN

PAT_ASSUM ``~p==>r`` (fn th => ASSUME_TAC (REWRITE_RULE [] (DISJ_IMP (ONCE_REWRITE_RULE [DISJ_SYM] (IMP_ELIM th))))) THEN

ASSUME_TAC(IMP_TRANS(ASSUME ``~r==>p``)(ASSUME ``p==>s``)) THEN

ASSUME_TAC (REWRITE_RULE [] (IMP_ELIM (ASSUME ``~r==>s``))) THEN

ASM_REWRITE_TAC []
);
\end{lstlisting}

\section{Session Transcript}


\setcounter{sessioncount}{0}
\begin{session}
  \begin{scriptsize}
\begin{verbatim}


> val problem3_thm =
   
[~((p :bool) /\ (q :bool)), ~(p :bool) ==> (r :bool),
 ~(q :bool) ==> (s :bool)] |- (r :bool) \/ (s :bool):
   thm
val it = (): unit
> 
\end{verbatim}
  \end{scriptsize}
\end{session}

%% ------------------------------------------
%% Change to letters for appendix
%% ------------------------------------------

%% ------------------------------------------
%% this restarts the section numbering
%% ------------------------------------------
\appendix{} 


%% ------------------------------------------
% label using capital letters
%% ------------------------------------------
\renewcommand{\thechapter}{\Alph{chapter}} 

\chapter{Source Code for exercise9Script}
\label{cha:source-code-sample}

The following code is from \emph{exercise9Script.sml}, which is located
in directory "../HOL/" 
\lstinputlisting{../HOL/exercise9Script.sml}

\chapter{Source Code for exercise10Script}
\label{cha:source-code-sample}

The following code is from \emph{exercise10Script.sml}, which is located
in directory "../HOL/"
\lstinputlisting{../HOL/exercise10Script.sml}

\end{document}
