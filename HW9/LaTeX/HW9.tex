%% ---------------------------------------------------
%% Notice that we use the "report" class instead of "article"
%% ---------------------------------------------------
\documentclass{report}

\title{Project 9}
\author{\textbf{Jinhao Wei}}
\date{\textbf{26 March 2019}}

%% ---------------------------------------------------
%% CSBDformat specifies the format of our reports
%% ---------------------------------------------------
\usepackage{634format}

%% ---------------------------------------------------
%% enumerate 
%% ---------------------------------------------------
\usepackage{enumerate}

%% ---------------------------------------------------
%% listings is used for including our source code in reports
%% ---------------------------------------------------
\usepackage{listings}
\usepackage{textcomp}

%% ---------------------------------------------------
%% Packages for math environments
%% ---------------------------------------------------
\usepackage{amsmath}

%% ---------------------------------------------------
%% Packages for URLs and hotlinks in the table of contents
%% and symbolic cross references using \ref
%% ---------------------------------------------------
\usepackage{hyperref}

%% ---------------------------------------------------
%% Packages for using HOL-generated macros and displays
%% ---------------------------------------------------
\usepackage{holtex}
\usepackage{holtexbasic}
\input{commands}

\begin{document}


\input{../HOL/HOLReports/HOLcipher}
\newcommand{\HOLcryptoExerciseDate}{26 March 2019}
\newcommand{\HOLcryptoExerciseTime}{16:58}
\begin{SaveVerbatim}{HOLcryptoExerciseTheoremsexerciseOneFiveXXSixXXFourXXOneaXXthm}
\HOLTokenTurnstile{} \HOLSymConst{\HOLTokenForall{}}\HOLBoundVar{key} \HOLBoundVar{enMsg} \HOLBoundVar{message}.
     (\HOLConst{deciphS} \HOLBoundVar{key} \HOLBoundVar{enMsg} \HOLSymConst{=} \HOLConst{SOME} \HOLBoundVar{message}) \HOLSymConst{\HOLTokenEquiv{}}
     (\HOLBoundVar{enMsg} \HOLSymConst{=} \HOLConst{Es} \HOLBoundVar{key} (\HOLConst{SOME} \HOLBoundVar{message}))
\end{SaveVerbatim}
\newcommand{\HOLcryptoExerciseTheoremsexerciseOneFiveXXSixXXFourXXOneaXXthm}{\UseVerbatim{HOLcryptoExerciseTheoremsexerciseOneFiveXXSixXXFourXXOneaXXthm}}
\begin{SaveVerbatim}{HOLcryptoExerciseTheoremsexerciseOneFiveXXSixXXFourXXOnebXXthm}
\HOLTokenTurnstile{} \HOLSymConst{\HOLTokenForall{}}\HOLBoundVar{keyAlice} \HOLBoundVar{k} \HOLBoundVar{text}.
     (\HOLConst{deciphS} \HOLBoundVar{keyAlice} (\HOLConst{Es} \HOLBoundVar{k} (\HOLConst{SOME} \HOLBoundVar{text})) \HOLSymConst{=}
      \HOLConst{SOME} \HOLStringLit{This is from Alice}) \HOLSymConst{\HOLTokenEquiv{}}
     (\HOLBoundVar{k} \HOLSymConst{=} \HOLBoundVar{keyAlice}) \HOLSymConst{\HOLTokenConj{}} (\HOLBoundVar{text} \HOLSymConst{=} \HOLStringLit{This is from Alice})
\end{SaveVerbatim}
\newcommand{\HOLcryptoExerciseTheoremsexerciseOneFiveXXSixXXFourXXOnebXXthm}{\UseVerbatim{HOLcryptoExerciseTheoremsexerciseOneFiveXXSixXXFourXXOnebXXthm}}
\begin{SaveVerbatim}{HOLcryptoExerciseTheoremsexerciseOneFiveXXSixXXFourXXTwoaXXthm}
\HOLTokenTurnstile{} \HOLSymConst{\HOLTokenForall{}}\HOLBoundVar{P} \HOLBoundVar{message}.
     (\HOLConst{deciphP} (\HOLConst{pubK} \HOLBoundVar{P}) \HOLFreeVar{enMsg} \HOLSymConst{=} \HOLConst{SOME} \HOLBoundVar{message}) \HOLSymConst{\HOLTokenEquiv{}}
     (\HOLFreeVar{enMsg} \HOLSymConst{=} \HOLConst{Ea} (\HOLConst{privK} \HOLBoundVar{P}) (\HOLConst{SOME} \HOLBoundVar{message}))
\end{SaveVerbatim}
\newcommand{\HOLcryptoExerciseTheoremsexerciseOneFiveXXSixXXFourXXTwoaXXthm}{\UseVerbatim{HOLcryptoExerciseTheoremsexerciseOneFiveXXSixXXFourXXTwoaXXthm}}
\begin{SaveVerbatim}{HOLcryptoExerciseTheoremsexerciseOneFiveXXSixXXFourXXTwobXXthm}
\HOLTokenTurnstile{} \HOLSymConst{\HOLTokenForall{}}\HOLBoundVar{key} \HOLBoundVar{text}.
     (\HOLConst{deciphP} (\HOLConst{pubK} \HOLFreeVar{Alice}) (\HOLConst{Ea} \HOLBoundVar{key} (\HOLConst{SOME} \HOLBoundVar{text})) \HOLSymConst{=}
      \HOLConst{SOME} \HOLStringLit{This is from Alice}) \HOLSymConst{\HOLTokenEquiv{}}
     (\HOLBoundVar{key} \HOLSymConst{=} \HOLConst{privK} \HOLFreeVar{Alice}) \HOLSymConst{\HOLTokenConj{}} (\HOLBoundVar{text} \HOLSymConst{=} \HOLStringLit{This is from Alice})
\end{SaveVerbatim}
\newcommand{\HOLcryptoExerciseTheoremsexerciseOneFiveXXSixXXFourXXTwobXXthm}{\UseVerbatim{HOLcryptoExerciseTheoremsexerciseOneFiveXXSixXXFourXXTwobXXthm}}
\begin{SaveVerbatim}{HOLcryptoExerciseTheoremsexerciseOneFiveXXSixXXFourXXThreeXXthm}
\HOLTokenTurnstile{} \HOLSymConst{\HOLTokenForall{}}\HOLBoundVar{signature}.
     \HOLConst{signVerify} (\HOLConst{pubK} \HOLFreeVar{Alice}) \HOLBoundVar{signature}
       (\HOLConst{SOME} \HOLStringLit{This is from Alice}) \HOLSymConst{\HOLTokenEquiv{}}
     (\HOLBoundVar{signature} \HOLSymConst{=}
      \HOLConst{sign} (\HOLConst{privK} \HOLFreeVar{Alice}) (\HOLConst{hash} (\HOLConst{SOME} \HOLStringLit{This is from Alice})))
\end{SaveVerbatim}
\newcommand{\HOLcryptoExerciseTheoremsexerciseOneFiveXXSixXXFourXXThreeXXthm}{\UseVerbatim{HOLcryptoExerciseTheoremsexerciseOneFiveXXSixXXFourXXThreeXXthm}}
\newcommand{\HOLcryptoExerciseTheorems}{
\HOLThmTag{cryptoExercise}{exercise15_6_4_1a_thm}\HOLcryptoExerciseTheoremsexerciseOneFiveXXSixXXFourXXOneaXXthm
\HOLThmTag{cryptoExercise}{exercise15_6_4_1b_thm}\HOLcryptoExerciseTheoremsexerciseOneFiveXXSixXXFourXXOnebXXthm
\HOLThmTag{cryptoExercise}{exercise15_6_4_2a_thm}\HOLcryptoExerciseTheoremsexerciseOneFiveXXSixXXFourXXTwoaXXthm
\HOLThmTag{cryptoExercise}{exercise15_6_4_2b_thm}\HOLcryptoExerciseTheoremsexerciseOneFiveXXSixXXFourXXTwobXXthm
\HOLThmTag{cryptoExercise}{exercise15_6_4_3_thm}\HOLcryptoExerciseTheoremsexerciseOneFiveXXSixXXFourXXThreeXXthm
}


%% --------------------------------------------------- the listings
%% parameter "language" is set to "ML"
%% ---------------------------------------------------
\lstset{language=ML,breaklines}


\maketitle{}

\begin{abstract}
  This report is basically a summary on my attempts on Project 9, which includes tactic-based proof of cryptography theorems. This report provides my solutions on \emph{exercise 15.6.1}, \emph{15.6.2} and \emph{15.6.3}. In addition, I had fine printed the corresponding datatypes and proofs and put the reports in \emph{../HOL/HOLReports/cryptoExerciseReport.pdf}.
\end{abstract}

\begin{acknowledgments}
  This project follows the format and structure of \emph{sampleTheory} provided by Professor Shiu-Kai Chin. To make it more accurate, this project mostly followed the format of one of my previous projects, which is project 5, and project 5 followed the sturcture of Professor Shiu-Kai Chin's \emph{sampleTheory} project. Besides, this project relies on \emph{../HOL/cipherScript.sml} and \emph{../HOL/isainfRules.sml}, which are also provided by Professor Shiu-Kai Chin.
\end{acknowledgments}

\tableofcontents{}


\chapter{Executive Summary}
\label{cha:executive-summary}

\textbf{All requirements for this project are satisfied}.  In
particular, we defined all the datatypes and proved all the theorems in this project, pretty printed the HOL theories,
and made use of the \emph{EmitTeX} structure to typeset HOL theorems
in this report.

We gave proofs on the following theorems:
\begin{quote}
  \HOLcryptoExerciseTheorems
\end{quote}

\begin{description}
\item [Reproducibility in ML and \LaTeX]\ \\
All ML and \LaTeX{} source files compile well on the environment provided by this course.
\end{description}


\chapter{Exercise 15.6.1}
\label{cha:e1161}

\section{Problem Statement}
\label{sec:e1161ps}
In this exercise, we gave proofs of two theorems:
\HOLcryptoExerciseTheoremsexerciseOneFiveXXSixXXFourXXOneaXXthm 

and 

\HOLcryptoExerciseTheoremsexerciseOneFiveXXSixXXFourXXOnebXXthm

Before we go through the following sections, we will need to run
\begin{lstlisting}[frame=trBL]
app load ["cipherTheory","stringTheory"]
open HolKernel Parse boolLib bossLib
open TypeBase isainfRules optionTheory
open cipherTheory
\end{lstlisting}
in HOL session.
\section{Proof of exercise15_6_1a_thm}
\label{sec:e1161definition}

In this section, we will prove
\HOLcryptoExerciseTheoremsexerciseOneFiveXXSixXXFourXXOneaXXthm

\subsection{Relevant Code}
\label{subsec:e1161code}
We used the following code to construct a goal-oriented proof.
\begin{lstlisting}[frame=trBL]
val exercise15_6_4_1a_thm =
TAC_PROOF (
([],
``!key enMsg message.(deciphS key enMsg = SOME message)= (enMsg = Es key (SOME message))``),
PROVE_TAC [deciphS_one_one]
);
\end{lstlisting}

\subsection{Session Transcript}
\label{subsec:e1161st}
If we send the above code to HOL, we will see the transcript as below:
\begin{session}
  \begin{scriptsize}
\begin{verbatim}
 
> # # # # # <<HOL message: inventing new type variable names: 'a>>
Meson search level: ..........
val exercise15_6_4_1a_thm =
   |- !(key :symKey) (enMsg :'a symMsg) (message :'a).
     (deciphS key enMsg = SOME message) <=>
     (enMsg = Es key (SOME message)):
   thm
\end{verbatim}
\end{scriptsize}
\end{session}


\section{Proof of exercise15_6_1b_thm}
\label{sec:e1161definition}
In this section, we will prove theorem \HOLcryptoExerciseTheoremsexerciseOneFiveXXSixXXFourXXOnebXXthm

\subsection{Relevant Code}
\label{subsec:e1161code}
We will use the following code to give a tacti-based proof.
\begin{lstlisting}[frame=trBL]
val exercise15_6_4_1b_thm =
TAC_PROOF(
([], ``!keyAlice k text.(deciphS keyAlice (Es k (SOME text)) = SOME "This is from Alice") = (k=keyAlice)/\(text = "This is from Alice")``),
PROVE_TAC [deciphS_one_one]
)
\end{lstlisting}

\subsection{Session Transcript}
\label{subsec:e1161st}
If we send the above code to HOL, we will see the transcript as below:
\begin{session}
  \begin{scriptsize}
\begin{verbatim}
 
> # # # Meson search level: ......................
val it =
   |- !(keyAlice :symKey) (k :symKey) (text :string).
     (deciphS keyAlice (Es k (SOME text)) =
      SOME "This is from Alice") <=>
     (k = keyAlice) /\ (text = "This is from Alice"):
   thm
\end{verbatim}
\end{scriptsize}
\end{session}



\chapter{Exercise 15.6.2}
\label{cha:e1161}

\section{Problem Statement}
\label{sec:e1161ps}
In this exercise, we gave proofs on two theorems
\HOLcryptoExerciseTheoremsexerciseOneFiveXXSixXXFourXXTwoaXXthm and
\HOLcryptoExerciseTheoremsexerciseOneFiveXXSixXXFourXXTwobXXthm

Before we go through the following sections, we will need to run
\begin{lstlisting}[frame=trBL]
app load ["cipherTheory","stringTheory"]
open HolKernel Parse boolLib bossLib
open TypeBase isainfRules optionTheory
open cipherTheory
\end{lstlisting}
in HOL session.

\section{Proof of exercise_15_6_2a_thm}
\label{sec:e1161definition}

In this section, we will prove
\HOLcryptoExerciseTheoremsexerciseOneFiveXXSixXXFourXXTwoaXXthm

\subsection{Relevant Code}
\label{subsec:e1161code}
We used the following code to construct a goal-oriented proof.
\begin{lstlisting}[frame=trBL]
val exercise15_6_4_2a_thm =
TAC_PROOF(
([], ``!P message. (deciphP (pubK P) enMsg = SOME message) = (enMsg = Ea (privK P)(SOME message))``),
PROVE_TAC [deciphP_one_one]
);
\end{lstlisting}

\subsection{Session Transcript}
\label{subsec:e1161st}
If we send the above code to HOL, we will see the transcript as below:
\begin{session}
  \begin{scriptsize}
\begin{verbatim}
 
> # # # # <<HOL message: inventing new type variable names: 'a, 'b>>
Meson search level: ..........
val exercise15_6_4_2a_thm =
   |- !(P :'a) (message :'b).
     (deciphP (pubK P) (enMsg :('b, 'a) asymMsg) = SOME message) <=>
     (enMsg = Ea (privK P) (SOME message)):
   thm
\end{verbatim}
\end{scriptsize}
\end{session}


\section{Proof of exercise15_6_2b_thm}
\label{sec:e1161definition}
In this section, we will prove theorem \HOLcryptoExerciseTheoremsexerciseOneFiveXXSixXXFourXXTwobXXthm

\subsection{Relevant Code}
\label{subsec:e1161code}
We will use the following code to give a tactic-based proof.
\begin{lstlisting}[frame=trBL]
val exercise15_6_4_2b_thm =
TAC_PROOF(
([],``!key text.(deciphP (pubK Alice) (Ea key (SOME text)) = SOME "This is from Alice") = (key = privK Alice)/\(text = "This is from Alice")``),
PROVE_TAC [deciphP_one_one]
)
\end{lstlisting}

\subsection{Session Transcript}
\label{subsec:e1161st}
If we send the above code to HOL, we will see the transcript as below:
\begin{session}
  \begin{scriptsize}
\begin{verbatim}
 
> # # # # <<HOL message: inventing new type variable names: 'a>>
Meson search level: ......................
val exercise15_6_4_2b_thm =
   |- !(key :'a pKey) (text :string).
     (deciphP (pubK (Alice :'a)) (Ea key (SOME text)) =
      SOME "This is from Alice") <=>
     (key = privK Alice) /\ (text = "This is from Alice"):
   thm
\end{verbatim}
\end{scriptsize}
\end{session}


\chapter{Exercise 15.6.3}
\label{cha:e1161}

\section{Problem Statement}
\label{sec:e1161ps}
In this exercise, we gave our proof on theorem

\HOLcryptoExerciseTheoremsexerciseOneFiveXXSixXXFourXXThreeXXthm
%{\UseVerbatim{HOLcryptoExerciseTheoremsexerciseOneFiveXXSixXXFourXXThreeXXthm}}

Before we go through the following sections, we will need to run
\begin{lstlisting}[frame=trBL]
app load ["cipherTheory","stringTheory"]
open HolKernel Parse boolLib bossLib
open TypeBase isainfRules optionTheory
open cipherTheory
\end{lstlisting}
in HOL session.

\section{Relevant Code}
\label{subsec:e1161code}
We used the following code to construct a goal-oriented proof.
\begin{lstlisting}[frame=trBL]
val exercise15_6_4_3_thm =
TAC_PROOF(
([], ``!signature. signVerify (pubK Alice) signature (SOME "This is from Alice") = (signature = sign(privK Alice)(hash (SOME "This is from Alice")))``),
PROVE_TAC [signVerify_one_one]
);
\end{lstlisting}

\section{Session Transcript}
\label{subsec:e1161st}
If we send the above code to HOL, we will see the transcript as below:
\begin{session}
  \begin{scriptsize}
\begin{verbatim}
 
> # # # # <<HOL message: inventing new type variable names: 'a>>
Meson search level: ..........
val exercise15_6_4_3_thm =
   |- !(signature :(string digest, 'a) asymMsg).
     signVerify (pubK (Alice :'a)) signature
       (SOME "This is from Alice") <=>
     (signature = sign (privK Alice) (hash (SOME "This is from Alice"))):
   thm
\end{verbatim}
\end{scriptsize}
\end{session}


%% ------------------------------------------
%% Change to letters for appendix
%% ------------------------------------------

%% ------------------------------------------
%% this restarts the section numbering
%% ------------------------------------------
\appendix{} 


%% ------------------------------------------
% label using capital letters
%% ------------------------------------------
\renewcommand{\thechapter}{\Alph{chapter}} 

\chapter{Source Code for cipherScript.sml}
\label{cha:source-code-sample}

The following code is from \emph{cipherScript.sml}, which is located
in directory "../HOL/" 
\lstinputlisting{../HOL/cipherScript.sml}

\chapter{Source Code for cryptoExerciseScript.sml}
\label{cha:source-code-sample}

The following code is from \emph{cryptoExerciseScript.sml}, which is located
in directory "../HOL/"
\lstinputlisting{../HOL/cryptoExerciseScript.sml}


\end{document}
